\documentclass[a4paper,10pt]{scrartcl}
\usepackage[margin=2cm,bindingoffset=0cm]{geometry}
\usepackage{ucs}
\usepackage[utf8x]{inputenc}
\usepackage[ngerman]{babel}
\usepackage{fontenc}
%\usepackage[pdftex]{graphicx}
\usepackage{listings}
\usepackage{amssymb}
\usepackage{amsmath}
\usepackage{wasysym}
\usepackage{graphicx}
\usepackage[pdftex]{hyperref}

\author{Verena Käfer (2551188), Niklas Schnelle (2573250), Peter Vollmer (2553704)}
\date{25.10.2010}
\title{Projektplan von OsmUi}

\begin{document}
\maketitle
\newpage
\tableofcontents
\newpage


\section{Einleitung}
\subsection{Zweck, Abgrenzung}
OsmUi stellt eine benutzerfroundlichere Schnittstelle zur Benutzung von Osmosis dar. Hierbei soll das Pipeline Modell, welches von Osmosis verwendet wird, auf ein leicht benutzbares GUI Modell übertragen werden.
OsmUi verarbeitet hierbei selbst keine Daten von OpenStreetMap, sondern generiert Aufrufe für Osmosis oder ruft dieses
selbst auf. 
\subsection{Projektüberblick, Motivation}
Da es sehr mühselig ist, die komplexen Pipline-Konstruktionen für Osmosis immer wieder per Hand auf der Kommandozeile oder in einer bat/sh Datei einzugeben, 
enstand die Idee einer grafischen Benutzeroberfläche für Osmosis.
Diese Idee wird nun in Form des Softwarepraktikums im Wintersemester 2010/11 an der Universität Stuttgart umgesetzt.
Eine der so entstehenden Lösungen ist OsmUi, folgende Grundfunktionalitäten sollen durch OsmUi zur Verfügung gestellt werden:
\begin{itemize}
\item Generierung einer Kommandozeile
\item Graphische Aufbereitung der Task-Auswahl und Konstruktion von Piplines
\end{itemize}

\section{Formale Grundlagen}
\subsection{Vertragliche Anforderungen an die Projektdurchführung}
Das Projekt muss bis zum 01.02.2010 fertiggestellt werden. Dabei werden pro Person 180 Arbeitsstunden veranschlagt.
\subsection{Vertragliche Anforderungen an das Projekt}
\begin{itemize}
\item OsmUi muss mit 100 Tasks gleichzeitig klarkommen und auf allen Plattformen laufen, die Java SE unterstützen.  
\end{itemize}
\section{Leistungen der Vertragspartner}
\subsection{Lieferumfang (Software, Dienstleistungen)}
Im Lieferumfang enthalten sind:
\begin{itemize}
\item OsmUi
\item Handbuch (online / PDF) in englischer Sprache
\item Code
\item Installationsanleitung
\end{itemize}
\subsection{Resultate, die nicht zum Lieferumfang gehören}
Testplan, Projektplan, Spezifikation
\subsection{Leistungen des Auftraggebers}
Leistungen des Auftraggebers:
\begin{itemize}
\item Bibliothek zum auswählen des Kartenausschnitts.
\item Bibliothek zum Preview der durch Osmosis generierten Ausgabe
\item kleines komplettes Beispiel für eine typische Osmosis Benutzung
\item einige Beispiel-Kommandozeilenaufrufe
\end{itemize}

\subsection{Externe Meilensteine}
\begin{tabular}{|c|c|p{14em}|p{14em}|}
\hline Datum & Uhrzeit & Beschreibung & Meilensteinbeauftragter\\ 
\hline 19.10.2010 & 15:45 - 17:15 Uhr & Kick-Off & Verena käfer, Niklas Schnelle, Peter Vollmer\\ 
\hline 29.10.2010 & 12:00 Uhr & Abgabe Analysenotizen + Projektplan & Peter Vollmer\\ 
\hline 16.11.20101 & 12:00 Uhr & Abgabe Spezifikation + UI Prototyp & Niklas Schnelle\\ 
\hline 30.11.2010 & 12:00 Uhr & Abgabe korrigierte Spezifikation + Zwischenstand Zeitabrechnung & Verena Käfer\\ 
\hline 14.12.2010 & 12:00 Uhr & Abgabe Entwurf & Peter Vollmer\\ 
\hline 31.12.2010 & 23:59 Uhr & Abgabe Zwischenstand Implementierung (Alpha) + Systemtestplan & Niklas Schnelle\\ 
\hline 11.01.2011 & 12:00 Uhr & Abgabe Implementierung (Beta) + Modultest & Verena Käfer\\ 
\hline 25.01.2011 & 12:00 Uhr & Abgabe Implementierung (RC) + Systemtestprotokoll & Peter Vollmer\\ 
\hline 01.02.2011 & ganzer Tag & Abnahme durch den Kunden & Verena käfer, Niklas Schnelle, Peter Vollmer\\ 
\hline 11.02.2011 & & Ende des Softwarepraktikums & Verena käfer, Niklas Schnelle, Peter Vollmer\\ 
\hline 
\end{tabular} 
\subsection{Abnahmeprozedur}
Die Abnahme erfolgt durch den Kunden, vertreten durch Igor Podolskiy und/oder Holger Röder.
\subsection{Änderungsverfahren}
Anforderungsänderungen werden vom Kunde über das Kundenforum bekannt gegeben. Vom Hersteller ausgehende Änderungswünsche werden mit dem
Kunden und den Betreuern abgesprochen.


\section{Entwicklungsprozess}
\subsection{Strategie für die Entwicklung und Integration}
Die Entwicklung erfolgt nach einem erweiterten Wasserfallmodell.
In der Implementierungsphase wird außerdem verstärkt Pair Programming eingesetzt werden.
Wobei die Entwicklung von einem Chief Programmer geleitet wird, wichtig ist hierbei, dass dieser die Verantwortung für die eigentliche Entwicklung,
angefangen beim Entwurf, tragen wird, aber keine Verwaltungsaufgaben, die über die Implementierung hinaus gehen, hat.
\subsection{Projektspezifische Abweichungen vom Standardprozess}

\subsection{Phasen der Entwicklung}
Die weitere Entwicklung erfolgt in  folgenden Phasen:
\begin{enumerate}
\item
Spezifikation und Spezifikationsreview
\item
Entwurf unter anderem mit UML Diagrammen
\item
Implementierung, sowohl des  eigentlichen Softwaresystems als auch der Unit Tests,
wobei bereits fertige Einheiten so früh wie möglich mit passenden Unit Tests getestet werden.
\item 
Test des Gesamtsystems und Korrektur von Fehlern
\item
Interne Abnahme bei erfolgreich bestandenen Tests des Gesamtsystems
\item
Abnahme durch den Kunden
\end{enumerate}
\subsection{Dokumentationsplan}
Folgende Dokumente werden erstellt und auf dem aktuellen Stand gehalten:
\begin{itemize}
\item Projektplan + Gantt- und Termin-Drift-Diagramm
\item Spezifikation
\item Handbuch/Online-Hilfe
\item Source/API Dokumentation mit Javadoc. Hierbei wird als Teil des Styleguides jede nach außen sichtbare (public) Methode und jede Klasse dokumentiert.
\end{itemize}

\section{Risiken}
\subsection{Risiken und ihre Bewertung}
\begin{itemize}
\item Serverabstürze/Datenverlust\\
Dieses Risiko ist hoch, wobei es meist durch Fehlbedienung und nicht durch Hardwareausfall auftritt.
\item Ausscheiden eines Teilnehmers aus dem Projekt\\
Dieses Risiko ist gering.
\item Temporärer Ausfall eines Teilnehmers aus dem Projekt.\\
Dieses Risiko ist hoch, da Krankheiten nicht vorhergesehen werden können.
\end{itemize}


\subsection{Maßnahmen zur Reduktion der Risiken}
\subsubsection{Datenverlust}
Um dem Verlust wichtiger Projektdaten vorzubeugen, wird git als dezentralisiertes Versionsverwaltungssystem eingesetzt.
So hat jeder Entwickler immer eine gesamte Kopie der Versionshistorie. Diese von jedem wieder hergestellt werden, sollte sie auf dem Projektserver oder bei einem der Entwickler verloren gehen.
\subsubsection{Wegfall}
Durch sorgfälltige Dokumentation und einer detailierten Spezifikation fällt es neuen Mitarbeitern leichter sich einzuarbeiten.
\subsubsection{Ausfall}
Durch das vermeiden unnötiger Risiken wird temporärer Ausfall minimiert.
\subsection{Notfallpläne}
Im Notfall an den Betreuer wenden.
\section{Richtlinien für die Entwicklung}
\subsection{Konfigurationsmanagement}
Als Konfirgurationsverwaltung wird git verwendet sowie als webbasiertes Projektmanagementtool Trac. Desweiteren existiert eine Projektinterne Mailingliste.
\subsection{Design- und Programmierrichtlinien}
Als Programmierrichtlinie wird die Java Code Convention von Oracle/Sun verwendet, siehe \url{http://www.oracle.com/technetwork/java/codeconvtoc-136057.htm}l
Als Dokumentationssystem wird Javadoc verwendet.
\subsection{Einsatz von Werkzeugen}
Als IDE wird Eclipse Helios genutzt, für Code Coverage Messungen das Plugin CodeCover. Hinzu kommen die unter Konfigurationsverwaltung genannten Werkzeuge sowie die Java Entwicklungswerkzeuge des Sun- und OpenJDK.
\section{Anforderungen an die Umgebung}
Die Software wird auf allen durch Java SE unterstützen Systeme lauffähig sein, denn es werden nur reine Java Bibliotheken eingesetzt.
%\subsection{Infrastruktur (Büros, Rechnersysteme, Software)}
%\subsection{Leistungen Dritter im Unternehmen}
%\subsection{Leistungen externer Lieferanten}

\section{Projektorganisation}
\subsection{Schnittstelle zum Auftraggeber}
Schnittstelle zum Kunden ist das Kundenforum. 
\subsection{Schlüsselpersonen}
\begin{itemize}
\item
Projektleiter : Peter Vollmer (vollmepr@studi.informatik.uni-stuttgart.de)
\item
Chiefprogrammer : Niklas Schnelle (schnelns@studi.informatik.uni-stuttgart.de)
\item
Chieftester : Verena Käfer (kaeferva@studi.informatik.uni-stuttgart.de
\item
externer Kunden : Igor Podolskiy\\
(\href{http://www.uni-stuttgart.de/iev/institut/php-Dateien/php_HompageAnzeigen.php?Name=Pd}{http://www.uni-stuttgart.de/iev/institut/php-Dateien/php\_HompageAnzeigen.php?Name=Pd})
\item
interner Kunde : Holger Röder\\
(\href{http://www.iste.uni-stuttgart.de/se/menschen/roeder.html}{http://www.iste.uni-stuttgart.de/se/menschen/roeder.html})
\item 
Betreuer : Daniel Kulesz\\
(\href{http://www.iste.uni-stuttgart.de/se/menschen/kulesz.html}{http://www.iste.uni-stuttgart.de/se/menschen/kulesz.html})
\end{itemize}
\subsection{Organisation (differenziert für die einzelnen Projektphasen)}
\subsection{Berichtwesen}

\section{Entwicklungsplan}
\subsection{Projektstrukturplan (Arbeitsgliederung)}
\subsection{Terminplan}
\includegraphics[width=15cm]{gantt.png}
\subsection{Kostenplan}
Es werden pro Person 180 Arbeitsstunden veranschlagt, dem Kunden entstehen aber keine Kosten.

\end{document}
