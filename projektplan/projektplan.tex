\documentclass[a4paper,10pt]{scrartcl}
\usepackage[margin=2cm,bindingoffset=1cm]{geometry}
\usepackage{ucs}
\usepackage[utf8x]{inputenc}
\usepackage[ngerman]{babel}
\usepackage{fontenc}
%\usepackage[pdftex]{graphicx}
\usepackage{listings}
\usepackage{amssymb}
\usepackage{amsmath}
\usepackage{wasysym}

\usepackage[pdftex]{hyperref}

\author{Verena Käfer (2551188), Niklas Schnelle (2573250), Peter Vollmer (2553704)}
\date{25.10.2010}
\title{Projektplan von OsmUi}

\begin{document}
\maketitle
\newpage
\tableofcontents
\newpage


\section{Einleitung}
\subsection{Zweck, Abgrenzung}
OsmUi stellt eine benutzerfreundlichere Schnittstelle zur Benutzung von Osmosis dar. Hierbei soll das Pipeline Modell, welches von Osmosis verwendet wird, auf ein leicht benutzbares GUI Modell übertragen werden. 
\subsection{Projektüberblick, Motivation}
Aus der schwierigen Aufgabe die Kommandozeilen für Osmosis zu generieren entstand die Idee für OsmUi. Dabei stellt OsmUi folgende Grundfunktionalitäten bereit:

- Generierung einer Kommandozeile
- Graphische Aufbereitung der Task-Auswahl



\section{Formale Grundlagen}
\subsection{Vertragliche Anforderungen an die Projektdurchführung}
Das Projekt muss bis zum 01.02.2010 fertiggestellt werden. Dabei werden pro Person 180 Arbeitsstunden veranschlagt.
\subsection{Vertragliche Anforderungen an das Produkt}
OsmUi muss mit 100 Tasks gleichzeitig klarkommen und auf allen Plattformen laufen, die Java se unterstützen. 


\section{Leistungen der Vertragspartner}
\subsection{Lieferumfang (Software, Dienstleistungen)}
Im Lieferumfang enthalten sind:
- OsmUi
- Handbuch (online / PDF) in englischer Sprache
- Code
- Installationsanleitung
\subsection{Resultate, die nicht zum Lieferumfang gehören}
Testplan, Projektplan, Spezifikation
\subsection{Leistungen des Auftragebers}
Leistungen des Auftraggebers:

-Application Interface zum auswählen des Kartenausschnitts.
-Application Interface zum Preview der gewählten Optionen.
-kleines komplettes Beispiel
-einige Beispiel-Kommandozeileneingaben
\subsection{Externe Meilensteine}
\begin{tabular}{|c|c|c|}
\hline Datum & Uhrzeit & Beschreibung\\ 
\hline 19.10.2010 & 15:45 - 17:15 Uhr & Kick-Off\\ 
\hline 29.10.2010 & 12:00 Uhr & Abgabe Analysenotizen + Projektplan\\ 
\hline 16.11.20101 & 12:00 Uhr & Abgabe Spezifikation + UI Prototyp\\ 
\hline 30.11.2010 & 12:00 Uhr & Abgabe korrigierte Spezifikation + Zwischenstand Zeitabrechnung\\ 
\hline 14.12.2010 & 12:00 Uhr & Abgabe Entwurf\\ 
\hline 31.12.2010 & 23:59 Uhr & Abgabe Zwischenstand Implementierung (Alpha) + Systemtestplan\\ 
\hline 11.01.2011 & 12:00 Uhr & Abgabe Implementierung (Beta) + Modultest\\ 
\hline 25.01.2011 & 12:00 Uhr & Abgabe Implementierung (RC) + Systemtestprotokoll\\ 
\hline 01.02.2011 & ganzer Tag & Abnahme durch den Kunden\\ 
\hline 11.02.2011 & & Ende des Softwarepraktikums\\ 
\hline 
\end{tabular} 
\subsection{Abnahmeprozedur}
Kunde versucht mit unserem Produkt zurrecht zu kommen.
\subsection{Änderungsverfahren}
Änderungen werden vom Kunde über das Kundenforum bekanntgegeben. Unsere Änderungen werden mit den Betreuern abgesprochen.


\section{Entwiklungsprozess}
\subsection{Strategie für die Entwicklung und Integration}
\subsection{Projektspezifische Abweichungen vom Standardprozess}
\subsection{Phasen der Entwiklung}
\subsection{Dokumentationsplan}


\section{Risiken}
\subsection{Risiken und ihre Bewertung}
\subsection{Maßnahmen zur Reduktion der Risiken}
\subsection{Notfallpläne}


\section{Richtlinien für die Entwiklung}
\subsection{Konfigurationsmanagement}
\subsection{Design- und Programmierrichtlinien}
\subsection{Einsatz von Werkzeugen}



\section{Anforderungen an die Umgebung}
\subsection{Infrastruktur (Büros, Rechnersysteme, Software)}
\subsection{Leistungen Dritter im Unternehmen}
\subsection{Leistungen externer Lieferanten}




\section{Projektorganisation}
\subsection{Schnittstelle zum Auftraggeber}
\subsection{Schnittstellen zur eigenen Organisation}
\subsection{Schnittstellen zu anderen Projekten}
\subsection{Schlüsselpersonen}
\subsection{Organisation (differenziert für die einzelnen Projektphasen)}
\subsection{Berichtwesen}


\section{Entwiklungsplan}
\subsection{Projektstrukturplan (Arbeitsgliederung)}
\subsection{Terminplan}
\subsection{Kostenplan}



\end{document}
