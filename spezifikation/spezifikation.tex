\documentclass[a4paper,10pt]{scrartcl}
\usepackage[margin=2cm,bindingoffset=0cm]{geometry}
\usepackage{ucs}
\usepackage[utf8x]{inputenc}
\usepackage[ngerman]{babel}
\usepackage{fontenc}
%\usepackage[pdftex]{graphicx}
\usepackage{listings}
\usepackage{amssymb}
\usepackage{amsmath}
\usepackage{wasysym}
\usepackage{graphicx}
\usepackage[pdftex]{hyperref}
\author{Verena Käfer (2551188), Niklas Schnelle (2573250), Peter Vollmer (2553704)}
\date{erstellt am 25.10.2010\\
Version: 1.0}
\title{\includegraphics[width=15cm]{../projektplan/Logo_Osmui.png} \\ 
Spezifikation von OsmUi}

\begin{document}
\maketitle
\newpage
\tableofcontents
\newpage

\section{Einleitung}
\subsection{Zweck der Spezifikation}
Bei der Spezifikation handelt es sich um das zentrale Projektdokument. Sie ist Basis aller weiteren Dokumente und
Bezugspunkt in Funktionalitätsfragen. Sie ist daher unbedingt aktuell und konsistent zu halten.\\
Die Spezifikation beschreibt die Funktionen und Anforderungen an das Produkt OsmUi und das zugehörige Projekt.
\subsection{Leserkreis}
Diese Spezifikation richtet sich an:
\begin{itemize}
 \item Die Entwickler von OsmUi
 \item Den Kunden und die Betreuer des SoPra
\end{itemize}

\subsection{Projektüberblick}
In dem Entwicklungsprojekt OsmUi soll eine grafische Benutzeroberfläche für das Kommandozeilenwerkzeug \href{http://wiki.openstreetmap.org/wiki/Osmosis}{Osmosis}
entwickelt werden. Zu diesem Zweck überträgt OsmUi das abstrakte Pipelinekonzept von Osmosis auf eine grafische Darstellung, welche die gesamte
Funktionalität von Osmosis zugänglich macht, dabei aber deutlich benutzerfreundlicher sein soll.\\
Während bei Osmosis ein langer Kommandozeilenaufruf eine Pipeline zur Verarbeitung von OpenStreetMap Daten beschreibt, wobei sich der Benutzer die Befehle merken
und korrekt in der gewünschten Reihenfolge aufschreiben muss, soll es mit OsmUi möglich sein eine Verarbeitungspipeline ``zusammen zu klicken''.\\
Hierbei kann die Funktionsweise der einzelnen Tasks sowie deren Interaktion der Benutzeroberfläche von OsmUi entnommen werden.
%\subsection{Konventionen} %Kommt noch

\section{Akteure}
\subsection{Benutzer}
Bei OsmUi gibt es grundsätzlich nur eine Nutzerklasse. Es wird davon ausgegangen, dass OsmUi benutzt wird um eine Verarbeitungspipeline für OpenStreetMap Daten
zu erstellen, welche anschließend durch Osmosis ``ausgeführt'' wird. Hierbei wird davon ausgegangen, dass der Nutzer grundlegende Kentnisse 
von Datenverarbeitung und OpenStreetMap hat.

\section{Nichtfunktionale Anforderungen}
\subsection{Entwurfseinschränkungen}
\subsubsection{Entwurfskonzept}
OsmUi wird nach dem \href{http://de.wikipedia.org/wiki/KISS-Prinzip}{KISS-Prinzip} entwickelt, d.h. entscheidend für die Qualität der Software soll
sein, wie gut die Hauptfunktionen unterstützt werden. Bessere Hauptfunktionalität ist einem größeren Funktionsumfang unterzuordnen.\\
Im Fall von OsmUi heißt dies, dass das Hauptaugenmerk auf dem einfachen und sinnvollen Erstellen von Verarbeitungspipelines, sowie deren Import und Export liegt.\\
Weniger wichtig sind hingegen ``Luxusfunktionen'' wie direktes Anzeigen des Verarbeitungsergebnisses (was auf Grund der Vielfalt der Funktionen von Osmosis
und der Verarbeitungsdauer ohnehin nicht immer sinnvoll ist) oder das direkte Aufrufen von Osmosis.
\subsubsection{Systemumgebung}
Systemumgebung ist das \textbf{Java Runtime Environment in Version 1.6 und höher (SE)}. Dabei ist OsmUi als reine Java Anwendung zu entwickeln, so dass
keine externen Abhängigkeiten bestehen und OsmUi auf allen Java SE kompatiblen Systemen eingesetzt werden kann.
Eventuell wird intern die Bibliothek \textbf{JGraph} eingesetzt, diese wird dann aber direkt mit ausgeliefert
und ist ebenso in platformunabhängigem Java geschrieben. 
\subsubsection{Layout und Gestaltung}
OsmUi soll ab einer Auflösung von 1024x768 benutzbar sein. Alle Fenster sollen, wenn dies sinnvoll ist, skalierbar sein und den eventuell gewonnenen Platz
nutzen. Um Benutzer von Multimonitorsystemen und exotischen Windowmanagern zu entlasten soll die Positionierung von neuen Fenstern nativ erfolgen. Es
sollen insbesondere keine Fenster in eine errechnete Bildschirmmitte von OsmUi positioniert werden.
\subsection{Robustheit}
OsmUi soll als Software möglichst robust arbeiten, d.h. unter allen Bedingungen korrekt arbeiten. Dies soll durch defensive Programmierung erreicht werden.
Jedoch sollen Benutzereingaben möglichst früh geprüft werden, so dass tiefere Funktionen den Daten ``vertrauen'' können. Dies sichert zudem ab, dass klar ist, welcher
Programmteil für die Prüfung zuständig ist. Nämlich immer der erste, der die nötigen Informationen hat.\\
Weiterhin wird die Robustheit durch das KISS-Prinzip unterstützt, da weniger Funktionen besser entworfen und getestet werden können.
\subsection{Portabilität}
OsmUi soll durch den Einsatz reinen Javas auf allen Systemen mit Java SE 1.6 Unterstützung lauffähig sein. Als Exportformate für Osmosis Aufrufskripte
werden .bat und .sh (/bin/sh) unterstützt, womit alle Posix Systeme sowie Windows abgedeckt sind.
\subsection{Erweiterbarkeit}
Das Programm muss keine besonderen Erweiterungsfunktionalitäten wie etwa ein Pluginsystem zur Verfügung stellen, jedoch werden der Programmcode und die Systemarchitektur
so gestaltet, dass OsmUi leicht und schnell an neue oder veränderte Osmosis Funktionen angepasst werden kann.
\section{Distributionsform und Installation}
OsmUi wird als ausführbares JAR Archiv ausgeliefert, welches direkt ausgeführt werden kann.


\section{Funktionale Anforderungen}
\subsection{Mengengerüst}
Bei OsmUi handelt es sich um eine Einzelplatzanwendung und es wird keine Netzwerkfunktion genutzt, somit gibt es zu jedem Zeitpunkt genau einen Nutzer.\\
Beim Erstellen/Laden/Speichern von Verarbeitungspipelines muss OsmUi in der Lage sein, mit Pipelines von bis zu 100 Tasks zuverlässig zu arbeiten -
eine künstliche Beschränkung nach oben besteht jedoch nicht.
\subsection{Laden und Speichern von Verarbeitungspiplines als Osmosis Aufrufscript}
Dem KISS-Prinzip folgend bietet OsmUi eine einheitliche Laden- und Speichern-Funktion, die sowohl bereits vorhandene Osmosis Aufrufscripts als auch durch OsmUi
selbst erstellte Verarbeitungspiplines, laden kann. Die Speicherung erfolgt als Osmosis Aufrufskripte.\\
Dies ermöglicht auch beim Wechsel des Werkzeugs einen leichten Zugriff auf alle mit OsmUi erstellten Dateien.
\subsubsection{Laden}
Eine der Hauptfunktionen von OsmUi stellt das Laden von Aufrufscripten dar. Dabei können sowohl Dateien im .bat - als auch im .sh Format (mit \#!/bin/sh Shebang),
sowie Textdateien, in denen nur eine Osmosis Parameterliste steht, geladen werden. Sie werden direkt im Pipelinebearbeitungsfeld angezeigt und
stehen zur Bearbeitung bereit.
\subsubsection{Speichern}
Zum Speichern wählt der Benutzer entweder einen neue Datei aus, die, wenn vorhanden, überschrieben und sonst neu erstellt wird. Oder er überschreibt die gerade bearbeitete Datei.
Der Dateityp kann hierbei zwischen .sh Script und .bat gewählt werden.
\subsection{Benutzeroberfläche}
Die Benutzeroberfläche von OsmUi ist in zwei Teile aufgeteilt: die Taskauswahlbox und die Pipelinebox. Desweiteren gibt es ein Anwendungsmenü und beim Bearbeiten eines Tasks
wird es ein weiteres Einstellungsfenster geben.
\subsubsection{Taskauswahlbox}
Im linken Teil der Benutzeroberfläche wird eine Liste der gerade zum Einfügen verfügbaren Tasks dargestellt. Dabei kann die Darstellung als Icon,
Text oder Kombination erfolgen. Es werden nur Tasks angezeigt die gerade auch einfügbar sind. \\
Am Anfang sind dies zum Beispiel alle Tasks, (z.B. read XML) die keinen Streameingang haben. Ist in der Pipelinebox ein Task ausgewählt, so
sind nur diejenigen Tasks sichtbar, die mit dem/den Ausgang/Ausgängen des ausgewählten Task kompatibel sind.\\
So wird verhindert, dass der Benutzer Pipelines bauen kann, die nicht durch Osmosis ausführbar sind.
\subsubsection{Pipelinebox}
In der Pipelinebox wird die Pipeline grafisch dargestellt und es können Tasks zum Bearbeiten ausgewählt werden. Ist ein Task ausgewählt so können seine Eigenschaften
bearbeitet oder ein neuer Task über die Taskauswahlbox - wie oben beschrieben - angefügt werden. Desweiteren kann eine Verbindung zu einem anderen Task, der noch offene
Inputs hat, hergestellt werden.
\subsubsection{Taskeinstellungsdialog}
Wählt der Benutzer einen Task aus, so stellt OsmUi einen Dialog zur Verfügung, in dem alle - für den jeweiligen Tasktyp vorhandenen - Parameter eingegeben werden können.
Hierbei kann die Eingabe eingeschränkt sein: als Spanne von Zahlen, als Kartenausschnitt oder als freier Text erfolgen. Dabei wird der für
den Datentyp beste Eingabemodus gewählt.
\subsubsection{Lokalisation}
OsmUi wird eine übersetzte Benutzeroberfláche für mindestens die Sprachen Deutsch und Englisch bieten. Dabei wird die aktuell verwendete
Sprache aus den hierfür vorgesehenen Umgebungsvariablen eingelesen, um sie ohne Interaktion durch den Benutzer in dessen System einzugliedern.

\end{document}
